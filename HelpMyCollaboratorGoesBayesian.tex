\documentclass[doc]{apa6}
\usepackage{wrapfig}
\usepackage{lmodern}
\usepackage{amssymb,amsmath}
\usepackage{ifxetex,ifluatex}
\usepackage{fixltx2e} % provides \textsubscript
\ifnum 0\ifxetex 1\fi\ifluatex 1\fi=0 % if pdftex
  \usepackage[T1]{fontenc}
  \usepackage[utf8]{inputenc}
\else % if luatex or xelatex
  \ifxetex
    \usepackage{mathspec}
  \else
    \usepackage{fontspec}
  \fi
  \defaultfontfeatures{Ligatures=TeX,Scale=MatchLowercase}
\fi
% use upquote if available, for straight quotes in verbatim environments
\IfFileExists{upquote.sty}{\usepackage{upquote}}{}
% use microtype if available
\IfFileExists{microtype.sty}{%
\usepackage{microtype}
\UseMicrotypeSet[protrusion]{basicmath} % disable protrusion for tt fonts
}{}
\usepackage{hyperref}
\hypersetup{unicode=true,
            pdftitle={Help, My Collaborator Goes Bayesian! Why And How To Apply Bayesian Data Analysis},
            pdfauthor={Wouter de Nooy},
            pdfkeywords={R; problems null hypothesis significance testing; Bayesian statistical
analysis;},
            pdfborder={0 0 0},
            breaklinks=true}
\urlstyle{same}  % don't use monospace font for urls
\usepackage{color}
\usepackage{fancyvrb}
\newcommand{\VerbBar}{|}
\newcommand{\VERB}{\Verb[commandchars=\\\{\}]}
\DefineVerbatimEnvironment{Highlighting}{Verbatim}{commandchars=\\\{\}}
% Add ',fontsize=\small' for more characters per line
\usepackage{framed}
\definecolor{shadecolor}{RGB}{248,248,248}
\newenvironment{Shaded}{\begin{snugshade}}{\end{snugshade}}
\newcommand{\KeywordTok}[1]{\textcolor[rgb]{0.13,0.29,0.53}{\textbf{#1}}}
\newcommand{\DataTypeTok}[1]{\textcolor[rgb]{0.13,0.29,0.53}{#1}}
\newcommand{\DecValTok}[1]{\textcolor[rgb]{0.00,0.00,0.81}{#1}}
\newcommand{\BaseNTok}[1]{\textcolor[rgb]{0.00,0.00,0.81}{#1}}
\newcommand{\FloatTok}[1]{\textcolor[rgb]{0.00,0.00,0.81}{#1}}
\newcommand{\ConstantTok}[1]{\textcolor[rgb]{0.00,0.00,0.00}{#1}}
\newcommand{\CharTok}[1]{\textcolor[rgb]{0.31,0.60,0.02}{#1}}
\newcommand{\SpecialCharTok}[1]{\textcolor[rgb]{0.00,0.00,0.00}{#1}}
\newcommand{\StringTok}[1]{\textcolor[rgb]{0.31,0.60,0.02}{#1}}
\newcommand{\VerbatimStringTok}[1]{\textcolor[rgb]{0.31,0.60,0.02}{#1}}
\newcommand{\SpecialStringTok}[1]{\textcolor[rgb]{0.31,0.60,0.02}{#1}}
\newcommand{\ImportTok}[1]{#1}
\newcommand{\CommentTok}[1]{\textcolor[rgb]{0.56,0.35,0.01}{\textit{#1}}}
\newcommand{\DocumentationTok}[1]{\textcolor[rgb]{0.56,0.35,0.01}{\textbf{\textit{#1}}}}
\newcommand{\AnnotationTok}[1]{\textcolor[rgb]{0.56,0.35,0.01}{\textbf{\textit{#1}}}}
\newcommand{\CommentVarTok}[1]{\textcolor[rgb]{0.56,0.35,0.01}{\textbf{\textit{#1}}}}
\newcommand{\OtherTok}[1]{\textcolor[rgb]{0.56,0.35,0.01}{#1}}
\newcommand{\FunctionTok}[1]{\textcolor[rgb]{0.00,0.00,0.00}{#1}}
\newcommand{\VariableTok}[1]{\textcolor[rgb]{0.00,0.00,0.00}{#1}}
\newcommand{\ControlFlowTok}[1]{\textcolor[rgb]{0.13,0.29,0.53}{\textbf{#1}}}
\newcommand{\OperatorTok}[1]{\textcolor[rgb]{0.81,0.36,0.00}{\textbf{#1}}}
\newcommand{\BuiltInTok}[1]{#1}
\newcommand{\ExtensionTok}[1]{#1}
\newcommand{\PreprocessorTok}[1]{\textcolor[rgb]{0.56,0.35,0.01}{\textit{#1}}}
\newcommand{\AttributeTok}[1]{\textcolor[rgb]{0.77,0.63,0.00}{#1}}
\newcommand{\RegionMarkerTok}[1]{#1}
\newcommand{\InformationTok}[1]{\textcolor[rgb]{0.56,0.35,0.01}{\textbf{\textit{#1}}}}
\newcommand{\WarningTok}[1]{\textcolor[rgb]{0.56,0.35,0.01}{\textbf{\textit{#1}}}}
\newcommand{\AlertTok}[1]{\textcolor[rgb]{0.94,0.16,0.16}{#1}}
\newcommand{\ErrorTok}[1]{\textcolor[rgb]{0.64,0.00,0.00}{\textbf{#1}}}
\newcommand{\NormalTok}[1]{#1}
\usepackage{graphicx,grffile}
\makeatletter
\def\maxwidth{\ifdim\Gin@nat@width>\linewidth\linewidth\else\Gin@nat@width\fi}
\def\maxheight{\ifdim\Gin@nat@height>\textheight\textheight\else\Gin@nat@height\fi}
\makeatother
% Scale images if necessary, so that they will not overflow the page
% margins by default, and it is still possible to overwrite the defaults
% using explicit options in \includegraphics[width, height, ...]{}
\setkeys{Gin}{width=\maxwidth,height=\maxheight,keepaspectratio}
\IfFileExists{parskip.sty}{%
\usepackage{parskip}
}{% else
\setlength{\parindent}{0pt}
\setlength{\parskip}{6pt plus 2pt minus 1pt}
}
\setlength{\emergencystretch}{3em}  % prevent overfull lines
\providecommand{\tightlist}{%
  \setlength{\itemsep}{0pt}\setlength{\parskip}{0pt}}
\setcounter{secnumdepth}{5}
% Redefines (sub)paragraphs to behave more like sections
\ifx\paragraph\undefined\else
\let\oldparagraph\paragraph
\renewcommand{\paragraph}[1]{\oldparagraph{#1}\mbox{}}
\fi
\ifx\subparagraph\undefined\else
\let\oldsubparagraph\subparagraph
\renewcommand{\subparagraph}[1]{\oldsubparagraph{#1}\mbox{}}
\fi

%%% Use protect on footnotes to avoid problems with footnotes in titles
\let\rmarkdownfootnote\footnote%
\def\footnote{\protect\rmarkdownfootnote}


  \title{Help, My Collaborator Goes Bayesian! Why And How To Apply Bayesian Data
Analysis}
    \author{Wouter de Nooy\textsuperscript{1,2}}
    \date{}
  
\shorttitle{Help, My Collaborator Goes Bayesian!}
\affiliation{
\vspace{0.5cm}
\textsuperscript{1} Amsterdam School of Communication Research ASCor\\\textsuperscript{2} University of Amsterdam}
\keywords{R; problems null hypothesis significance testing; Bayesian statistical analysis;\newline\indent Word count: 6755}
\usepackage{csquotes}
\usepackage{upgreek}
\captionsetup{font=singlespacing,justification=justified}

\usepackage{longtable}
\usepackage{lscape}
\usepackage{multirow}
\usepackage{tabularx}
\usepackage[flushleft]{threeparttable}
\usepackage{threeparttablex}

\newenvironment{lltable}{\begin{landscape}\begin{center}\begin{ThreePartTable}}{\end{ThreePartTable}\end{center}\end{landscape}}

\makeatletter
\newcommand\LastLTentrywidth{1em}
\newlength\longtablewidth
\setlength{\longtablewidth}{1in}
\newcommand{\getlongtablewidth}{\begingroup \ifcsname LT@\roman{LT@tables}\endcsname \global\longtablewidth=0pt \renewcommand{\LT@entry}[2]{\global\advance\longtablewidth by ##2\relax\gdef\LastLTentrywidth{##2}}\@nameuse{LT@\roman{LT@tables}} \fi \endgroup}


\geometry{a4paper, top=30mm, bottom = 25mm, asymmetric = FALSE}
\usepackage{float}

\authornote{Wouter de Nooy, Department of Communication
Science, University of Amsterdam.

Correspondence concerning this article should be addressed to Wouter de
Nooy, Nieuwe Achtergracht 166, 1018 WV Amsterdam. E-mail:
\href{mailto:w.denooy@uva.nl}{\nolinkurl{w.denooy@uva.nl}}}

\abstract{
This paper introduces researchers to a Bayesian approach to statistical
analyses. Starting with the problems of null hypothesis significance
testing, it outlines some fundamental advantages of a Bayesian
statistical analysis. Two types of Bayesian approaches are illustrated
using two different software packages: JASP and the R package
\texttt{rstanarm}.


}

\begin{document}
\maketitle

\section{What Is The Problem With Null Hypothesis Significance
Testing?}\label{what-is-the-problem-with-null-hypothesis-significance-testing}

Imagine that you have executed a regression analysis and one of the
regression coefficients is not statistically significant, so you do
\emph{not} reject the null hypothesis of no effect in the population. I
am nearly sure that you commit a \emph{Type II error} here: rejecting a
false null hypothesis. Why? Because the regression coefficient may be
very close to zero in the population but it will hardly ever be exactly
zero. So the null hypothesis of zero effect is nearly always false
(Bakan,
\protect\hyperlink{ref-bakanTestSignificancePsychological1966}{1966}:
425-6).

Why test a two-sided null hypothesis if we know that it is wrong? What
information do we gain if we reject a null hypothesis that is nearly
sure to be wrong? It is silly. From this perspective, it makes more
sense to test \emph{one-sided null hypotheses}. If we hypothesize that
an effect can only be positive, for example, the null hypothesis of no
effect or a negative effect can actually be true. In this case, we can
fail to reject the null hypothesis without making a Type II error.
Unfortunately, one-sided tests are unpopular in the field.

In a one-sided or two-sided test, can we conclude that the effect in the
population is more different from zero, that is, stronger, for a
rejected false null hypothesis than for a false null hypothesis that is
not rejected? No, we can't for several reasons. Our sample can be more
or less representative of the population: We may have drawn a sample
with a large effect from a population with a very small effect or the
other way around. In addition, a larger sample yields statistically
significant results more easily than a smaller sample. Very small
effects can be statistically significant in very large samples, whereas
large effects can be statistically non-significant in a very small
sample.

It is a common mistake to interpret statistical significance as an
indication that an effect is substantial or substantively interesting.
This mistake was noted early on (for example, by Tyler,
\protect\hyperlink{ref-tylerWhatStatisticalSignificance1931}{1931}). The
book \emph{Understanding the New Statistics} (Cumming,
\protect\hyperlink{ref-CummingUnderstandingnewstatistics2012}{2012})
offers a lucid discussion of this mistake. The binary decision
associated with a null hypothesis significance test incorrectly suggests
a qualitative difference between a statistically significant effect and
an effect that is not statistically significant.

It is easy to misunderstand and misinterpret p values and confidence
intervals because the probabilities involved are counterintuitive. We
would like to receive probabilities for population values, such as the
probability that the effect is positive in the population or the
probability that the population effect is within a particular range.
Instead, we get probabilities of sample outcomes. Instead of asking for
the probability that it will rain tomorrow, this approach reasons:
\enquote{If it rains tomorrow, what is the probability that we are
having at least as many clouds as we have today?}

The frequentist approach to statistics, which underlies null hypothesis
significance testing, assumes that there is one true population value.
In this approach, a population characteristic (parameter) does
\emph{not} have a probability. There is no probability distribution to a
parameter. As a consequence, the following statements are not correct:

\begin{itemize}
\tightlist
\item
  A p value is the probability that the null hypothesis is true.
\item
  A 95\%-confidence interval contains the population values that have
  95\% probability of being true.
\end{itemize}

The practical value of this approach is quite limited. Imagine that you
want to use your sample results for a practical decision such as
fielding a new campaign. Based on your sample, you would like to
calculate the probability that the effect of the campaign is above a
particular threshold, for example, the minimum effect size that makes
the investments in the campaign worthwhile. Your significance test only
tells you whether you should reject the null hypothesis of no effect.
Your confidence interval gives you plausible population values but not
the probabilities of those values. It doesn't tell you the probability
of a positive effect. Why? Because there is no probability distribution
for a population characteristic in this approach.

Assuming (in the null hypothesis) a true population value, very
efficient methods have been invented for generalizing results from
samples to populations. In the pre-computer and early-computer era, this
was the only option for statistical generalization. Thanks to our
present-day computing power, however, we have alternatives, such as
Bayesian statistics.

There are more problems with null hypothesis significance testing, some
of which will be touched upon in later sections of this paper. Let us
conclude this section with some references. You may want to know more
about these problems or your reviewer may ask you why you do not test a
null hypothesis. Consult or refer to publications like the
\emph{Understanding the New Statistics} book referenced above, (Ziliak
\& McCloskey,
\protect\hyperlink{ref-ziliakCultStatisticalSignificance2008}{2008}) or
(Morrison \& Henkel,
\protect\hyperlink{ref-morrisonSignificanceTestControversy1970}{1970})
for an older survey of the discussion. Carver
(\protect\hyperlink{ref-carverCaseStatisticalSignificance1993}{1993})
offers some suggestions on how to get articles published without
significance tests. Levine et al.
(\protect\hyperlink{ref-LevineQuantitativeCommunicationResearch2013}{2013};
\protect\hyperlink{ref-LevineCriticalAssessmentNull2008}{2008}) survey
the problems of null hypothesis significance testing in published
communication research.

\section{What Are The Benefits Of A Bayesian
Approach?}\label{what-are-the-benefits-of-a-bayesian-approach}

This paper does not pretend to present an introduction to Bayesian
statistics. It won't even present Bayes' theorem, which really is
inadmissible in an introduction. There are several introductions to
Bayesian statistics (Berry,
\protect\hyperlink{ref-BerryStatisticsBayesianPerspective1996}{1996};
Clark, \protect\hyperlink{ref-ClarkBayesianBasicsConceptual2014}{2014};
Kurt, \protect\hyperlink{ref-kurtBayesianStatisticsFun2019}{2019};
Lambert, \protect\hyperlink{ref-lambertStudentsGuideBayesian2018}{2018};
Lee,
\protect\hyperlink{ref-leeBayesianStatisticsIntroduction2012}{2012}),
among which I particularly recommend Erickson
(\protect\hyperlink{ref-EricksonBeginningBayes2017}{2017}) as a first
read. Here, I will just point out some benefits of a Bayesian approach
in comparison to the standard frequentist approach.

\subsection{Intuitive results}\label{intuitive-results}

In Bayesian statistics, population values have a probability
distribution. In other words, Bayesian statistics allow us to speak of
the probability that it will rain tomorrow given our knowledge of
today's weather. It gives us probabilities like the ones that we use in
ordinary life.

\begin{figure}
\centering
\includegraphics{HelpMyCollaboratorGoesBayesian_files/figure-latex/posteriorexample-1.pdf}
\caption{\label{fig:posteriorexample}An example of a probability
distribution for a parameter such as the regression coefficient in the
population.}
\end{figure}

In the example of a regression coefficient estimating the effect of
campaign exposure, a Bayesian analysis gives us a probability
distribution of the effect in the population. Practically speaking, we
obtain a long list of regression coefficient estimates, which is similar
to a bootstrapped distribution. Usually, the median value (vertical line
in Figure \ref{fig:posteriorexample}) of this distribution is used as
the point estimate.

The point estimate is a very poor summary of the distribution. The
probabilities implied by the distribution are more interesting. The
proportion of the estimates that are above zero, for example, can be
interpreted as the probability that the effect is positive in the
population. If you like, you can use this probability to test the
one-sided hypothesis that the effect is positive. We can actually
conclude that the probability of a positive effect in the population is
98 per cent in this example. In contrast, a significance test of the
null hypothesis that the effect is not negative only allows us to
conclude that our sample is very likely \emph{if the null hypothesis is
true}. We cannot reject the null hypothesis but we don't know the
probability of a positive effect in the population.

A probability distribution of the regression coefficient in the
population helps us to understand that it does not make sense to focus
on one value as in a two-sided null hypothesis significance test. In
theory, a regression coefficient, like most statistics, can have an
infinite number of decimal places. The probability of one exact value
such as 0.000000000000000000000(\ldots{}and so on) must be close to
zero. In other words, this one value is highly improbable. We know that
beforehand, so we are wasting our time if we test against this one
value.

Instead, we have to look at a range of values for the statistic
(Kruschke,
\protect\hyperlink{ref-kruschkeBayesianAssessmentNull2011}{2011}: 302),
the regression coefficient in the current example, if we want to use a
probability in our conclusion. In line with the concept of confidence
intervals, Bayesians use the \emph{credible interval} as the population
values that are most likely to be true. The 95\% credible interval, for
example, contains the 95 per cent estimates in the center of the
probability distribution. Here, we can say that the population value has
95 per cent probability (or chance) to be within this credible interval.
This is how we would like to (and sometimes erroneously do) interpret
the traditional 95 per cent confidence interval.

Let us return to the practical decision on launching a campaign based on
a pretest in a sample. The Bayesian probability distribution of the
regression coefficient for the effect of campaign exposure tells us the
probability of a positive effect versus the probability of a negative
effect in the population. If the campaign is intended to have a positive
effect, we may advise to launch the campaign if the probability of a
positive effect is much larger than the probability of a negative
effect. What constitutes \enquote{much larger} is a substantive
decision, not a statistical decision. If we can quantify the costs and
benefits of the campaign, we can use the probability distribution to
calculate the expected (averaged over the probability distribution)
benefits and compare them to the costs.

\subsection{Other distributional assumptions}\label{assumptions}

The sampling distribution is central to null hypothesis significance
testing and confidence intervals. We usually approximate the sampling
distribution by a theoretical probability distribution, such as the
normal, t, or F distribution. The approximation only works if the data
meet requirements such as a normal distribution in the population, equal
population variances, or minimum sample size. If your data does not meet
the requirements, the results are not to be trusted.

The sampling distribution, however, does not play a role in a Bayesian
approach. In the latter approach, we are estimating probabilities of
population values, not probabilities of samples. As a consequence, we do
not care about requirements for approximating the sampling distribution
with a theoretical probability distribution. The probability
distribution resulting from a Bayesian analysis can have any shape.

This freedom comes at a cost. In Bayesian analysis, the researcher has
to specify a probability distribution for each population value that is
to be estimated. This probability distribution is called the \emph{prior
distribution}. The sample data update the prior distribution into the
\emph{posterior distribution}, which is the probability distribution
that the researcher interprets and reports. A Bayesian approach, then,
comes with a different type of distributional assumptions, namely the
choice of prior distributions.

How do we chose prior distributions? In the ideal case, the posterior
distributions of previous research can be used as prior distributions in
the analysis of new data. This is called an \emph{informative prior
distribution}. It represents more or less established knowledge about
the statistic that we estimate, which should not change much if it is
updated by the data in our new sample. This is a wonderful way of using
existing information and accumulating knowledge.

If we do not have this type of information, we may resort to
\emph{weakly informative prior distributions}. These are theoretical
probability distributions like the normal or t distribution centered at
a possible and not too unlikely value. As prior distributions, they
express our belief that the population value of the statistic is most
likely to be found in a relatively restricted range. But we do not rule
out that the true values are much higher or much lower than expected.

These prior distributions are called weakly informative because they are
easily overridden by the sample data if the sample is not very small. As
a consequence, the posterior distribution is usually hardly influenced
by the choice of a prior distribution. A plot of a weakly informative
prior distribution and the posterior distribution should show that the
two are quite different.

It is relatively safe to chose a weakly informative prior distribution.
Still, it is wise to check that the results do not change substantially
if you change the prior settings (robustness check). For guidelines, see
Depaoli and van de Schoot
(\protect\hyperlink{ref-depaoliImprovingTransparencyReplication2017}{2017}){]},
who recommend to compare with a model estimated from \emph{uninformative
prior distributions}, which more or less give equal prior probabilities
to all possible values of the statistic in the population.

In addition, a Bayesian approach requires another type of check. The
posterior distribution is usually estimated with an iterative (MCMC)
algorithm. This algorithm requires many steps (iterations) to arrive at
a stable level of the parameter estimate. This phase is called the
burn-in or warm-up phase. The algorithm is said to have converged when
the estimates stabilize within a fixed range. When convergence has been
achieved, an additional set of steps is required to find and count the
values that the parameter estimate can take.

The estimates from the steps in this second phase constitute the
posterior distribution of the parameter, which is the analysis result.
Posterior distributions can only be trusted if the algorithm has
converged. Convergence can be checked visually or numerically; examples
are provided in Section \ref{usebayesian}. If these checks give rise to
doubts about convergence, the posterior distributions should not be
interpreted. The model may have problems, for example, it can be too
complicated to be estimated from the available data, or we simply need
more step (iterations) to achieve convergence.

Both phases usually require thousands of steps, so estimation may take
quite some computer time for complicated latent variable or multilevel
models. For complicated models, a computer with ample memory and a fast
processor comes in handy. Cloud computing is another option. For common
statistical models such as analysis of variance or linear regression,
however, ordinary computers produce results quickly.

Much work has be done and is still being done to develop safe weakly
informative priors for different types of models. But Bayesian
estimation is more complicated than approximating a sampling
distribution with a theoretical probability distribution. Especially for
complicated models, things may go wrong. For an extensive checklist, see
\emph{Improving transparency and replication in Bayesian statistics: The
WAMBS-Checklist} (Depaoli \& van de Schoot,
\protect\hyperlink{ref-depaoliImprovingTransparencyReplication2017}{2017}).
The upside: Complicated models can be estimated.

\subsection{Estimation of complicated
models}\label{estimation-of-complicated-models}

As stated in the preceding section, the prior distribution brings our
prior knowledge to bear in the estimation process. This knowledge can be
quite superficial, for example, we just know that very large effects are
impossible or highly unlikely, so we specify a weakly informative prior
distribution. Why would we want to take quite superficial prior
knowledge as our starting point?

A very practical reason for including this knowledge is that it helps
the estimation algorithm. The prior information nudges the estimation
algorithm in a plausible direction---even if it is not necessarily the
right direction---so estimation will converge more easily and more
quickly. For relatively simple models, this nudge hardly makes a
difference, but complicated models, such as multilevel models with
random intercepts and random slopes, can sometimes only be estimated
with the help of these nudges.

\subsection{Precautions against overfitting the sample
data}\label{precautions-against-overfitting-the-sample-data}

A focus on statistical significance is at the expense of attention to
model fit: How well does the model fit the current data? More
importantly, how well would the model fit new data? Would the results
replicate?

Traditional frequentist statistics gives us the point estimate that fits
the current sample best. For example, we get the value of the regression
coefficient that best predicts the dependent variable scores from the
independent variable cores in our sample. Another sample will most
probably be different, so our current regression coefficient estimate
will not be the best fitting estimate for the new sample. It is
overfitted to our current sample.

The problem of overfitting a model to the data increases if the model
becomes more complex. Every additional predictor adds to overfitting the
model to the sample. The more complicated the model, the higher the risk
of overfitting the model to the sample at hand. More complicated models,
however, are more realistic because empirical reality tends to be
complex. We end up with a conundrum, which is known as the bias-variance
trade-off (James, Witten, Hastie, \& Tibshirani,
\protect\hyperlink{ref-JamesIntroductionStatisticalLearning2017}{2017}:
34-5): We value realistic, hence complex models (little bias) but we
also value replication, which is compromised by overfitting the model to
the current sample (variance).

How can we identify and correct for overfitting? It is important to
realize that p values are not related to the probability that results
will replicate (e.g., see R. Carver,
\protect\hyperlink{ref-carverCaseStatisticalSignificance1978}{1978}:
385-6). Adjusted \emph{R}\textsuperscript{2} for a regression model
downsizes the proportion of variance predicted (\enquote{explained}) by
the model in proportion to the number of predictors used, but the
estimated regression coefficients are still overfitted to the sample
data.

We need another way to tell us the probability that our results will
replicate. The best option is to actually replicate our study. If
sufficient data are at hand, a split-half approach can be used, which
fits a model to the first half of the data (training set) and then tests
it against the second half of the data by calculating how well the
fitted model predicts the dependent variable in the test set (validation
set). With insufficient data for a split-half approach or if we do not
want to ignore half of the information when fitting and validating the
model, resampling techniques can be used. These techniques repeatedly
sample from the original sample and fit the model to each new sample
(cross-validation), for instance, leave one randomly chosen observation
out or leave out a randomly selected subset of observations (k-fold
cross-validation) (Efron \& Hastie,
\protect\hyperlink{ref-efronComputerAgeStatistical2016}{2016}: 213ff).

A Bayesian approach estimates a probability distribution (posterior) for
a population value, not the single value produced in a frequentist
approach. If we use the posterior distribution to predict outcomes in a
new or re-sampled data set, the predictions will generally be better
than using the point estimate obtained in a frequentist approach. There
are additional techniques for reducing overfitting in a Bayesian
approach, some of which are briefly discussed in the next section.

\section{How Can I Use A Bayesian Approach?}\label{usebayesian}

Statistical software for Bayesian analysis is increasingly available.
For example, Bayesian alternatives to many types of statistical models
are available as R packages; consult the task view
\href{https://cran.r-project.org/web/views/Bayesian.html}{Bayesian} or
search for \enquote{Bayes} in other
\href{https://cran.r-project.org/web/views/}{task views}.

Here, I focus on two software packages for Bayesian analysis: a
menu-driven software package (\emph{JASP}) with a work-flow like SPSS
for the most common statistical models and a command-driven R package
(\texttt{rstanarm}), which is very good at estimating complicated
models, such as non-linear multilevel models. The two software packages
represent two different Bayesian approaches: a focus on null hypothesis
testing versus a focus on parameter estimation (Kruschke,
\protect\hyperlink{ref-kruschkeBayesianAssessmentNull2011}{2011}).

\subsection{JASP}\label{jasp}

\href{https://jasp-stats.org/}{JASP} is free statistical software
developed and maintained at the University of Amsterdam by Eric-Jan
Wagenmakers. It offers an SPSS-like interface featuring both traditional
frequentist analyses and Bayesian analyses. In contrast to SPSS, the
output is very much like a real document, which can be edited, so the
user can add comments, interpretations, and conclusions. If you open a
data set, the last saved output document is also opened. Results tables
are formatted in APA style and they can be copied and pasted into Word
or LaTeX documents. Figures can be copied and exported and, finally,
there are provisions for interacting with the Open Science Framework. It
takes very little time to learn using this interface if you are used to
SPSS.

\begin{figure}[H]
\includegraphics[width=1\linewidth]{JASP} \caption{Annotated (red) screen shot of the JASP interface after executing a Bayesian and a traditional independent-samples t test.}\label{fig:JASPscreenshot}
\end{figure}

Traditional (frequentist) estimation is available for all standard tests
(t tests, ANOVA, regression, chi-squared tests) as well as more advanced
tests, namely, MANOVA, repeated measurements ANOVA, exploratory factor
analysis, confirmatory factor analysis for one factor, SEM (provided
that you can specify the model in \href{http://lavaan.ugent.be/}{lavaan}
syntax), meta analysis, and basic network analysis. Note that the SEM
menu has a separate entry for parallel mediation models, which does not
require writing syntax. It has an option to draw the mediation model,
although this figure does not include the estimate of the indirect
effect.

At the time of writing this text (November 2019), a Bayesian approach is
mainly available for comparing group averages, for example, using data
from experiments. JASP offers Bayesian analysis for t tests on one or
two means, ANOVA, Repeated measures ANOVA, and ANCOVA. In addition,
Bayesian estimation is available for correlations and for linear
regression but not for regression models with dichotomous or categorical
predictors (although the beta-version \emph{BAIN} module can compare
models with categorical predictors). Finally, binomial and multinomial
tests, contingency tables, and log-linear models can be executed in a
Bayesian way.

JASP's Bayesian approach focuses on \emph{Bayes factors}. A Bayes factor
quantifies the support in the sample for one model in comparison to
another model. The more a model is supported rather than the other
model, the more evidence we have for the former model. Model comparison
with Bayse factors takes into account the model complexity as a guard
against overfitting the model to the sample data (Kass \& Raftery,
\protect\hyperlink{ref-kassBayesFactors1995}{1995}). Rules of thumb have
been developed for interpreting Bayes factor values as strength of the
evidence in favor of a model. Check out the
\href{https://en.wikipedia.org/wiki/Bayes_factor\#Interpretation}{Wikipedia
page on the Bayes factor} for rules of thumb.

One of the two models that are compared is usually the model of no
effect representing the null hypothesis. Thus, Bayes factors offer an
alternative to traditional null hypothesis significance tests. A Bayes
factor gives us a value expressing how much better the model allowing
for an effect (H\textsubscript{1}) is supported than the model of the
null hypothesis (indicated by \emph{BF\textsubscript{10}} in JASP
output). At the same time, however, we can see how much better the null
hypothesis is supported by the data (\emph{BF\textsubscript{01}} in JASP
output). In Figure \ref{fig:JASPscreenshot}, the null hypothesis of
equal population averages is more strongly supported by the data than
the alternative hypothesis. In contrast, a null hypothesis significance
test only tells us whether or not we should reject the null hypothesis.
It doesn't provide us with evidence to accept it.

Bayes factors for a comparison between models representing a null and
alternative hypothesis depend on the prior distribution that is used
(Kruschke,
\protect\hyperlink{ref-kruschkeBayesianAssessmentNull2011}{2011}). The
choice of prior can make a world of difference here. Therefore, it is
strongly recommended to execute a sensitivity test: How does the choice
of prior distribution affect the Bayes factor? This type of test is
available in JASP as the \emph{Bayes factor robustness test} option.

\subsection{\texorpdfstring{R \texttt{rstanarm}
package}{R rstanarm package}}\label{r-rstanarm-package}

The other Bayesian approach focuses on parameter estimation, modelling,
and prediction. Here, we are usually dealing with more complicated
regression models: non-linear and multilevel models. Estimation of these
models is not trivial, requiring attention and effort. Estimation
results and problems help us to reflect on the nature of the phenomena
that we are trying to model. Null hypothesis testing plays, if any, a
minor role in this approach. The book \emph{Statistical Rethinking}
(McElreath,
\protect\hyperlink{ref-McElreathStatisticalRethinkingBayesian2015}{2015})
offers an accessible introduction to this Bayesian approach and
\emph{Bayesian Data Analysis} (Gelman et al.,
\protect\hyperlink{ref-GelmanBayesianDataAnalysis2013}{2013}) a more
technical introduction.

The focus here is on the R
\href{http://mc-stan.org/rstanarm/index.html}{\texttt{rstanarm} package}
for Bayesian applied regression modelling (Gabry \& Goodrich,
\protect\hyperlink{ref-R-rstanarm}{2019}) and some auxiliary packages.
This package was developed by Andrew Gelman's team at Columbia
University. It combines the power of the stan modelling language with
the ease of applying regression models offered by R packages such as
\texttt{lme4} (Bates, Maechler, Bolker, \& Walker,
\protect\hyperlink{ref-R-lme4}{2019}). In other words, you can run a
Bayesian regression model with syntax that is not more complicated than
running the same regression model in a frequentist approach. But yes,
you have to work with R.

Let us illustrate how to apply Bayesian regression models using
\texttt{rstanarm}. Consult (Muth, Oravecz, \& Gabry,
\protect\hyperlink{ref-TQMP14-2-99}{2018}) for a more detailed
step-by-step tutorial. Our example data concern students in a US
dormitory (Madan, Cebrian, Lazer, \& Pentland,
\protect\hyperlink{ref-MadanSocialsensingepidemiological2010}{2010}),
who submitted daily health reports including their sadness status (sad
versus not sad, variable \texttt{sad.depressed}) during a few months.
The data set can be downloaded
\href{https://wdenooy.github.io/Switch2Bayesian/sadness.csv}{here}. We
will try to predict their sadness status from the (logarithm of the)
number of contacts with sad friends within the dormitory using spatial
contacts (variable \texttt{prox\_24\_friend}) and mediated contacts
(variable \texttt{media\_24\_friend}) in the 24 hours preceding health
report submission. We control for the student's sadness status on the
previous day (variable \texttt{prev\_sad}). Does sadness spread via
contacts and communication?

\subsubsection{Estimation}\label{estimation}

Because of the dichotomous nature of the dependent variable, we use
logistic regression. We have multiple health reports per student, so we
use a multilevel model with health reports nested within students. As a
first step, we estimate a variance components model, that is, a model
with random (varying) intercepts but without random (varying) slopes and
student-level predictors (code not reproduced here). A random (varying)
intercept captures a student's propensity for sadness. Some students are
just more often sad than other students.

Next, we estimate a model with varying slopes of spatial contact (code
not reproduced here). This model allows spatial contact to have
different effects on the odds of being sad for different students.
Finally, we add two student-level predictors, namely popularity as the
(square root of) the number of friendship nominations received within
the dorm (variable \texttt{friends\_received\_sqrt}) and the student's
study phase (variable \texttt{phase\_adv}, 1 : advanced study phase, 0:
early study phase). The latter student-level variable is used to predict
variation in the effect (slopes) of spatial contact with sad friends.
The R code for this model is depicted below.

\begin{Shaded}
\begin{Highlighting}[]
\CommentTok{# R code for estimating a multilevel logistisc regression model with random}
\CommentTok{# intercepts and random slopes with student characteristic 'adv_phase'}
\CommentTok{# predicting slope variation.}

\CommentTok{# Load the rstanarm and rstan packages and set stan options.}
\KeywordTok{library}\NormalTok{(rstanarm)}
\KeywordTok{library}\NormalTok{(rstan)}
\CommentTok{#Set options for multicore CPU with excess RAM.}
\KeywordTok{rstan_options}\NormalTok{(}\DataTypeTok{auto_write =} \OtherTok{TRUE}\NormalTok{)}
\KeywordTok{options}\NormalTok{(}\DataTypeTok{mc.cores =}\NormalTok{ parallel}\OperatorTok{::}\KeywordTok{detectCores}\NormalTok{())}

\CommentTok{# Load the data (first download it to your R working directory).}
\NormalTok{reports <-}\StringTok{ }\KeywordTok{read.csv}\NormalTok{(}\StringTok{"sadness.csv"}\NormalTok{)}

\CommentTok{# Estimate the model.}
\NormalTok{reports_fit3 <-}\StringTok{ }\KeywordTok{stan_glmer}\NormalTok{(}\DataTypeTok{formula =}\NormalTok{ sad.depressed }\OperatorTok{~}\StringTok{  }\CommentTok{#outcome}
\StringTok{  }\NormalTok{prev_sad }\OperatorTok{+}\StringTok{ }\KeywordTok{log}\NormalTok{(media_24_friend}\OperatorTok{+}\DecValTok{1}\NormalTok{) }\OperatorTok{+}\StringTok{ }\KeywordTok{log}\NormalTok{(prox_24_friend}\OperatorTok{+}\DecValTok{1}\NormalTok{) }\OperatorTok{+}\StringTok{ }\CommentTok{#level-1 predictors}
\StringTok{  }\NormalTok{phase_adv }\OperatorTok{+}\StringTok{ }\NormalTok{friends_received_sqrt }\OperatorTok{+}\StringTok{ }\CommentTok{#level-2 predictors (for intercept)}
\StringTok{  }\NormalTok{phase_adv }\OperatorTok{*}\StringTok{ }\KeywordTok{log}\NormalTok{(prox_24_friend}\OperatorTok{+}\DecValTok{1}\NormalTok{) }\OperatorTok{+}\StringTok{ }\CommentTok{#level-2 predictor for slope}
\StringTok{  }\NormalTok{(}\DecValTok{1} \OperatorTok{+}\StringTok{ }\KeywordTok{log}\NormalTok{(prox_24_friend}\OperatorTok{+}\DecValTok{1}\NormalTok{) }\OperatorTok{|}\StringTok{ }\NormalTok{user_id), }\CommentTok{#intercept varying across students}
  \DataTypeTok{data =}\NormalTok{ reports, }\CommentTok{#data frame stored in sadness.csv}
  \DataTypeTok{family =} \KeywordTok{binomial}\NormalTok{(}\DataTypeTok{link =} \StringTok{"logit"}\NormalTok{), }\CommentTok{#logistic link}
  \CommentTok{# prior_intercept = normal(0,10,autoscale=TRUE), #intercept prior}
  \CommentTok{# prior = normal(0, c(2.5, 2.5, 2.5),autoscale=TRUE), #regression coefficients}
  \CommentTok{# prior_covariance = decov(regularization = 1, concentration = 1, shape = 1,}
  \CommentTok{# scale = 1), #covariance prior}
  \CommentTok{# iter = 2000, #default number of iterations (including warmup iterations)}
  \CommentTok{# warmup = floor(iter/2), #default number of warmup runs: half of all iterations}
  \CommentTok{# adapt_delta = 0.95, #default target average acceptance probability}
  \CommentTok{# max_treedepth = 10, #max steps within an iteration}
  \DataTypeTok{seed =} \DecValTok{9122}\NormalTok{) }\CommentTok{#randomiser seed for reproducibility}
\end{Highlighting}
\end{Shaded}

The \texttt{stan\_glmer()} function in the \texttt{rstanarm} package
mimics the \texttt{glmer()} function in the \texttt{lme4} package, which
estimates \textbf{g}eneral \textbf{l}inear \textbf{m}odels with a
multilevel structure (\textbf{m}ixed \textbf{e}ffects). The regression
model is specified as a formula with the dependent variable to the left
and the predictors to the right of the tilde. Random (varying)
intercepts and varying slopes are specified between parentheses using
the symbol \texttt{\textbar{}}, which separates the varying parameters
(\texttt{1} represents the intercept) from the level at which they are
allowed to vary (\texttt{user\_id} is the student ID number). The blog
\href{https://rpsychologist.com/r-guide-longitudinal-lme-lmer}{Using R
and lme/lmer to fit different two- and three-level longitudinal models}
(Magnusson, \protect\hyperlink{ref-magnussonUsingLmeLmer2015}{2015})
(Magnusson, \protect\hyperlink{ref-magnussonUsingLmeLmer2015}{2015})
shows how to specify more complicated multilevel models.

The \texttt{family\ =} argument is followed by arguments that are
specific to Bayesian inference. They are commented out to indicate that
they represent the default values at the time of writing this paper. The
default values for the prior are good starting values, but you would
have to change them if you want to compare the results to a model with
another prior (robustness check). Note that the default values may
change in new releases of the \texttt{rstanarm} package due to new
insights. For research to be reproducible, it is recommended to specify
the priors. You can get the default prior distributions if you apply the
\texttt{prior\_summary()} function to the fitted model.

The \texttt{iter\ =} argument is the most important argument for us. It
specifies the number of samples (iterations) that are used to estimate
the posterior distributions. On a computer with a multiple core
processor and with the option
\texttt{options(mc.cores\ =\ parallel::detectCores())} set, this number
of samples will be drawn by each available core. With a processor
containing four cores, four times 2,000 samples will be drawn by
default. Half of these are used for burning in (option
\texttt{warmup\ =}), that is, for converging to a stable range for the
parameter estimates. These samples are not used for the posterior
distribution. The second half of the samples are used to estimate the
parameters (the posterior distributions). If the estimation process does
not converge, the main solution is to increase the number of iterations.

\subsubsection{Convergence checks}\label{checks}

As argued in Section \ref{assumptions}, it is of paramount importance
that convergence has been achieved before we interpret and use the
posterior distributions. The R package \texttt{shinystan} (Gabry,
\protect\hyperlink{ref-R-shinystan}{2018}) offers an interactive user
interface for checking the estimation process and inspecting the
results.

\begin{Shaded}
\begin{Highlighting}[]
\CommentTok{# Package for interactive inspection of stan estimation process and results.}
\KeywordTok{library}\NormalTok{(shinystan) }

\CommentTok{# Display convergence checks for the third fitted (rstanarm) model.}
\CommentTok{# May take some time!}
\KeywordTok{launch_shinystan}\NormalTok{(reports_fit3)}
\end{Highlighting}
\end{Shaded}

\begin{figure}[H]
\includegraphics[width=1\linewidth]{shinystan1} \caption{Trace plot and posterior distribution for each parameter. Annotations are in red.}\label{fig:shinystanshot1}
\end{figure}

Shinystan opens a window in your web browser with four pages: Diagnose,
Estimate, Explore, and More. The Diagnose tab contains convergence
checks and other checks on the estimation process. Figure
\ref{fig:shinystanshot1} shows the first tab on the Diagnose page,
displaying the estimation process for the intercept of the regression
model.

The top-left graph is the \emph{trace plot}, showing the intercept
estimates (vertical axis) for all subsequent samples (iterations,
horizontal axis) for all cores (chains, represented by line color).
These are the estimates for the second half of the iterations, so after
warm-up/burn-in has completed. The estimation process must have
converged in the first half, so the average estimated value for the
intercept should not change (increase or decrease) during the second
half of the estimation process. This seems to be the case here. In
addition, the variation in estimates (vertical differences) should have
stabilized. Again, this seems to be the case here.

The top-right figure shows the posterior distribution of the intercept.
This is actually the analysis result, which is summarized on the
Estimate page. It is shown here because it should display a rather
regular, smooth shape. If we have too few iterations, this histogram
will look jagged.

These are just checks on the intercept. The drop-down list under
\emph{Parameter} allows us to check the other parameters. It is good
practice to check all parameters. A deviant pattern may indicate that a
parameter could not be estimated in a satisfactory way. Perhaps we just
have insufficient data to estimate a particular parameter or this
parameter is superfluous. In the present example, the variance of the
random slopes of spatial contact (parameter
\texttt{Sigma{[}user\_id:log(prox\_24\_friend\ +\ 1),log(prox\_24\_friend\ +\ 1)})
seems to be hard to estimate (if you run the code, you can check this).

\begin{figure}[H]
\includegraphics[width=1\linewidth]{shinystan2} \caption{Summary diagnostics for the sampler. Annotations are in red.}\label{fig:shinystanshot2}
\end{figure}

Figure \ref{fig:shinystanshot2} shows the \emph{HMC/NUTS (stats)} tab on
the Diagnose page. It summarizes some characteristics of the sampler
(\textbf{N}o \textbf{U}-\textbf{T}urn \textbf{S}ampler) used. I
recommend to select the \emph{Max} option and pay attention to the
\emph{treedepth} and \emph{divergent} columns. Tree depth represents the
number of steps within an iteration (sample). Maximum tree depth is 10
by default. If this maximum appears in this summary of sampler
parameters, it is wise to re-estimate the model with a higher value for
the \texttt{max\_treedepth\ =} argument.

The \emph{divergent} column should only contain zeros. Divergence during
the estimation process is a serious problem (see
\href{https://mc-stan.org/misc/warnings.html\#divergent-transitions-after-warmup}{Brief
Guide to Stan's Warnings}). Sometimes this problem can be solved by
setting the argument \texttt{adapt\_delta\ =} closer to 1.00. This will
increase estimation time.

\begin{figure}[H]
\includegraphics[width=1\linewidth]{shinystan3} \caption{Success of the sampler for parameters in terms of the standard error of the mean of the posterior draws, the effective number of samples, and potential scale reduction. Annotations are in red.}\label{fig:shinystanshot3}
\end{figure}

It can be more or less difficult for the sampler to find valid new
values for a parameter. It can be difficult if iterations are highly
correlated. If the sampler has substantial problems in this department,
relatively few values are found that can be regarded as truly different
samples. As a consequence, the number of effective samples (\(n_{eff}\))
is low, which increases the standard error of the mean of the estimates
(\(mcse\)). In this situation, the posterior distribution is based on
relatively little information. More iterations are needed to create a
reliable posterior distribution.

In addition, the Rhat (\(\hat{R}\)) statistic indicates if estimation
had converged. If convergence has been achieved, the value of this
statistic is near 1.0. A value close to 1.0 does not prove that
convergence has been achieved, but a value that is not close to 1.0
indicates convergence problems. In the latter case, we had better
re-estimate the model with a larger number of iterations or with a
larger share of the iterations as warm-up phase.

Parameters for which estimation is problematic according to the rules of
thumb (which can be adjusted in the right-hand panel) are listed below
the graphs. In Figure \ref{fig:shinystanshot3}, it only says
\enquote{none} below the graphs, so there are no parameters with
estimation problems according to these three checks.

\begin{figure}[H]
\includegraphics[width=1\linewidth]{shinystan4} \caption{Posterior predictive checks.}\label{fig:shinystanshot4}
\end{figure}

If the estimation process is successful, the estimated model should
predict the observed outcome scores well. We can use the (joint)
posterior distributions of the model parameters to predict outcome
scores. The PPcheck tab of the Diagnose page applies this principle many
times and calculates the mean and standard deviation of the outcome
scores. The results are represented by the light-blue histograms in
Figure \ref{fig:shinystanshot4}. In the present example, mean outcome is
the proportion or probability of reporting sadness. The dark vertical
line represents the proportion of sad reports in the sample. This value
should be somewhere near the center of the predicted values (the
histogram). This is the case here.

In addition to the mean, the standard deviation, minimum and maximum
value of the outcome variable are predicted and compared to the values
in the sample. In a logistic regression model, these values are not very
interesting. The standard deviation of a dichotomous variable (like sad
versus non-sad) is a function of the mean. The minimum and maximum
values can only be zero or one (if the outcome variable is coded this
way). For other types of regression models, however, these statistics
can be of interest.

\begin{figure}[H]
\includegraphics[width=1\linewidth]{shinystan5} \caption{Visual summary of the results.}\label{fig:shinystanshot5}
\end{figure}

The Estimate tab presents the estimation results both as a graph and a
table. The plot in Figure \ref{fig:shinystanshot5} summarizes the
posterior distributions of parameters. By default, the median of the
posterior distribution is used as the point estimate (blue dot) and both
the 50 and 95 per cent credible intervals are shown as a fat and thin
line segment. These settings can be adjusted with \emph{Posterior
interval} and \emph{Show/Hide Options}. The latter dialog also allows
color adjustment and exporting the graph as a \texttt{ggplot2} plot,
which can be included and further modified in R, or as a PDF picture.

The plot shows the first (twelve) model parameters but you can select
the parameters to be displayed. This does not work well in all browsers.
Also note that the web page may become unresponsive if you select many
parameters.

\begin{figure}[H]
\includegraphics[width=1\linewidth]{shinystan6} \caption{Tabular summary of the results.}\label{fig:shinystanshot6}
\end{figure}

The table of parameter estimates (Fig. \ref{fig:shinystanshot6}) can be
browsed and adapted more easily than the graphical representation.
Select the number of digits, columns, and rows (entries) to be shown.
Note that at most 100 rows can be shown. The table as shown can be
copied, printed, or downloaded as CSV or PDF. A separate tab on the
Estimate page offers the option to construct a LaTeX results table,
which can be copied and pasted into a (La)TeX word processor.

\subsubsection{Using the results}\label{using-the-results}

The main result of this type of Bayesian analysis consists of a (long)
list of estimated regression equations. Each iteration (sample) during
the estimation phase yields an estimate for each parameter in the model.
Table \ref{tab:jointposterior} shows a small part of this list.

\begin{Shaded}
\begin{Highlighting}[]
\CommentTok{# Display part of the joint posterior distribution.}

\CommentTok{# Requires the rstanarm, knitr and kableExtra libraries.}
\KeywordTok{library}\NormalTok{(rstanarm)}
\KeywordTok{library}\NormalTok{(knitr)}
\KeywordTok{library}\NormalTok{(kableExtra)}

\CommentTok{# Extract the parameter estimates from a fitted rstanarm model.}
\NormalTok{posterior_model3 <-}\StringTok{ }\KeywordTok{data.frame}\NormalTok{(reports_fit3, }\DataTypeTok{check.names =} \OtherTok{FALSE}\NormalTok{)}

\CommentTok{# Show the first 5 estimates for the (seven) fixed effects.}
\KeywordTok{kable}\NormalTok{(posterior_model3[}\DecValTok{1}\OperatorTok{:}\DecValTok{5}\NormalTok{, }\DecValTok{1}\OperatorTok{:}\DecValTok{7}\NormalTok{], }\DataTypeTok{digits =} \DecValTok{2}\NormalTok{, }\DataTypeTok{booktabs =} \OtherTok{TRUE}\NormalTok{,}
      \DataTypeTok{caption =} \StringTok{"The estimated fixed effects for the first five entries }
\StringTok{      in the joint posterior distribution of the model with random slopes }
\StringTok{      for spatial proximity and study phase as slope predictor."}\NormalTok{) }\OperatorTok
\StringTok{  }\KeywordTok{kable_styling}\NormalTok{(}\DataTypeTok{bootstrap_options =} \KeywordTok{c}\NormalTok{(}\StringTok{"basic"}\NormalTok{, }\StringTok{"condensed"}\NormalTok{), }
                \DataTypeTok{full_width =}\NormalTok{ T, }\DataTypeTok{font_size =} \DecValTok{12}\NormalTok{, }
                \DataTypeTok{latex_options =} \KeywordTok{c}\NormalTok{(}\StringTok{"scale_down"}\NormalTok{))}
\end{Highlighting}
\end{Shaded}

\begin{table}[H]

\caption{\label{tab:jointposterior}The estimated fixed effects for the first five entries 
      in the joint posterior distribution of the model with random slopes 
      for spatial proximity and study phase as slope predictor.}
\centering
\resizebox{\linewidth}{!}{
\fontsize{12}{14}\selectfont
\begin{tabular}[t]{rrrrrrr}
\toprule
(Intercept) & prev\_sad & log(media\_24\_friend + 1) & log(prox\_24\_friend + 1) & phase\_adv & friends\_received\_sqrt & log(prox\_24\_friend + 1):phase\_adv\\
\midrule
-5.63 & 2.29 & 0.29 & -0.24 & 0.01 & 0.68 & 0.35\\
-5.94 & 1.88 & 0.21 & -0.45 & -0.09 & 0.77 & 0.57\\
-7.29 & 1.96 & 0.17 & -0.41 & -0.63 & 1.30 & 0.51\\
-6.23 & 1.88 & 0.07 & -0.33 & -0.01 & 0.90 & 0.55\\
-7.18 & 2.24 & -0.19 & -0.18 & 0.42 & 1.27 & 0.35\\
\bottomrule
\end{tabular}}
\end{table}

The posterior distribution of a parameter is just the column of this
parameter in the list described above. The results presented by
\texttt{shinystan} (Section \ref{checks}) calculate the median (or mean)
as point estimate and the 2.5 and 97.5 percentiles as limits of the
credible interval. It is easy to calculate the probability that a
regression coefficient is larger than a particular value, for example,
that the effect of mediated exposure to sad friends is larger than zero:

\begin{Shaded}
\begin{Highlighting}[]
\KeywordTok{mean}\NormalTok{(posterior_model3}\OperatorTok{$}\StringTok{`}\DataTypeTok{log(media_24_friend + 1)}\StringTok{`} \OperatorTok{>}\StringTok{ }\DecValTok{0}\NormalTok{)}
\end{Highlighting}
\end{Shaded}

The list is a joint posterior distribution, meaning that you can also
extract probabilities for combinations of parameters. For example, the
probability that the effect of mediated exposure is larger (more
positive or less negative) than the effect of exposure through spatial
proximity is calculated as:

\begin{Shaded}
\begin{Highlighting}[]
\KeywordTok{mean}\NormalTok{(posterior_model3}\OperatorTok{$}\StringTok{`}\DataTypeTok{log(media_24_friend + 1)}\StringTok{`} \OperatorTok{>}\StringTok{ }
\StringTok{       }\NormalTok{posterior_model3}\OperatorTok{$}\StringTok{`}\DataTypeTok{log(prox_24_friend + 1)}\StringTok{`}\NormalTok{)}
\end{Highlighting}
\end{Shaded}

The result is 0.95.

In a similar way, we can find out more about the effect of spatial
proximity to sad friends for students who are in an advanced study
phase. Due to the interaction effect between study phase and spatial
proximity in the model, the estimated effect of spatial proximity on
sadness is the effect for the reference group scoring zero on study
phase. This is the effect for students in an early study phase, who are
coded as zero on variable \texttt{adv\_phase}.

The effect of spatial proximity for students in an advanced study phase
is the sum of the effect for early-phase students
(\texttt{log(prox\_24\_friend\ +\ 1)}) and the interaction effect
(\texttt{log(prox\_24\_friend\ +\ 1)}). Is this effect more likely to be
positive or negative? The probability that the sum of the two effects is
positive, is 0.86. This is calculated with the statement below:

\begin{Shaded}
\begin{Highlighting}[]
\KeywordTok{mean}\NormalTok{(posterior_model3}\OperatorTok{$}\StringTok{`}\DataTypeTok{log(prox_24_friend + 1)}\StringTok{`} \OperatorTok{+}\StringTok{ }
\StringTok{       }\NormalTok{posterior_model3}\OperatorTok{$}\StringTok{`}\DataTypeTok{log(prox_24_friend + 1):phase_adv}\StringTok{`} \OperatorTok{>}\StringTok{ }\DecValTok{0}\NormalTok{)}
\end{Highlighting}
\end{Shaded}

Finally, the posterior distribution can be used to visualize the
uncertainty about a regression coefficient. Just sample a number of
entries from the joint posterior distribution and use the regression
coefficien and the intercept to draw a regression line for each entry.
Figure \ref{fig:regressionplot} shows the median regression line (point
estimate) and 100 sampled regression lines for the effect of exposure
via spatial proximity for advanced-level students. Note that the plot
shows regression lines for an average student, not taking into account
variation across students modeled by random intercepts and slopes.

\begin{figure}
\centering
\includegraphics{HelpMyCollaboratorGoesBayesian_files/figure-latex/regressionplot-1.pdf}
\caption{\label{fig:regressionplot}The median regression line (dark blue)
and 100 sampled regression lines (light blue) for the effect of exposure
via spatial proximity for advanced-level students. Other predictors are
set to zero.}
\end{figure}

\subsubsection{Model comparison and
weighting}\label{model-comparison-and-weighting}

In the \texttt{rstanarm} approach to Bayesian analysis, model comparison
serves to optimize predictions on new data sets rather than establishing
the evidence for two competing hypotheses. If the simultaneous use of
competing models improves predictive accuracy, why stick to one model
and discard the other models (McElreath,
\protect\hyperlink{ref-McElreathStatisticalRethinkingBayesian2015}{2015}:
201)? If we acknowledge that we cannot be certain that one model is the
correct model, we had better not stick to one model.

Uncertainty about models is expressed by Akaike weights. Akaike weights
compare the information criterion scores of two or more models,
assigning weights to models that sum to one over all compared models. If
we want to predict new data using a set of models (a \emph{model
ensemble}), we weight the predictions by the model's Akaike weight.
Akaike model weights can be interpreted as the probability that a model
makes the best predictions within the set of compared models (Burnham \&
Anderson,
\protect\hyperlink{ref-BurnhamKullbackLeiblerinformationbasis2001}{2001};
Wagenmakers \& Farrell,
\protect\hyperlink{ref-WagenmakersAICmodelselection2004}{2004}).

This approach to model comparison, then, uses information criteria
rather than Bayes factors. A well-known information criterion is
Akaike's Information Criterion (AIC) (Hirotuge Akaike,
\protect\hyperlink{ref-AkaikeInformationtheoryextension1973}{1973};
Hirotugu Akaike,
\protect\hyperlink{ref-akaikeNewLookStatistical1974}{1974}). In Bayesian
analysis, however, other information criteria are preferred because they
make no distributional assumptions like AIC. Examples are the Widely
Applicable Information Criterion or Watanabe--Akaike Information
Criterion (WAIC) (Watanabe,
\protect\hyperlink{ref-WatanabeAsymptoticEquivalenceBayes2010}{2010})
and the Leave-One-Out cross-validation Information Criterion (LOOIC),
which is recommended over WAIC (Vehtari, Gelman, \& Gabry,
\protect\hyperlink{ref-Vehtari2017}{2017}).

The R code below illustrates the use of Leave-One-Out cross-validation
Information Criterion (LOOIC) and Akaike weights using this information
criterion for the three models predicting sadness introduced above.

\begin{Shaded}
\begin{Highlighting}[]
\CommentTok{# Model comparison using LOOIC and Akaike weights.}

\CommentTok{# Assumes that 3 rstanarm models have been fitted and saved as reports_fit1,}
\CommentTok{# reports_fit2, and reports_fit3.}

\CommentTok{# Requires the loo library.}
\KeywordTok{library}\NormalTok{(loo)}

\CommentTok{# Create LOO objects for all three models.}
\CommentTok{# If the models contain problematic observations, the k_threshold argument deals}
\CommentTok{# with those by re-estimating the model. This can be time consuming!}
\NormalTok{reports_loo1 <-}\StringTok{ }\KeywordTok{loo}\NormalTok{(reports_fit1, }\DataTypeTok{k_threshold =} \FloatTok{0.7}\NormalTok{)}
\NormalTok{reports_loo2 <-}\StringTok{ }\KeywordTok{loo}\NormalTok{(reports_fit2, }\DataTypeTok{k_threshold =} \FloatTok{0.7}\NormalTok{)}
\NormalTok{reports_loo3 <-}\StringTok{ }\KeywordTok{loo}\NormalTok{(reports_fit3, }\DataTypeTok{k_threshold =} \FloatTok{0.7}\NormalTok{)}

\CommentTok{# Model comparison.}
\CommentTok{# Tabulate LOOIC results for the three models.}
\NormalTok{reports_compare <-}\StringTok{ }\KeywordTok{loo_compare}\NormalTok{(reports_loo1, reports_loo2, reports_loo3)}

\CommentTok{# Calculate Akaike model weights.}
\NormalTok{reports_weights <-}\StringTok{ }\KeywordTok{loo_model_weights}\NormalTok{(}\KeywordTok{list}\NormalTok{(reports_loo1, reports_loo2, reports_loo3))}
\end{Highlighting}
\end{Shaded}

\begin{table}[H]

\caption{\label{tab:modelcomparison1}Three models arranged in descending order of predictive accuracy according to the approximate leave-one-out cross-validation information criterion (LOOIC).}
\centering
\resizebox{\linewidth}{!}{
\fontsize{12}{14}\selectfont
\begin{tabular}[t]{lrrrrrrrr}
\toprule
  & elpd\_diff & se\_diff & elpd\_loo & se\_elpd\_loo & p\_loo & se\_p\_loo & looic & se\_looic\\
\midrule
Plus student-level predictors & 0.0 & 0.0 & -463.4 & 27.8 & 41.4 & 4.2 & 926.8 & 55.6\\
Plus varying slope spatial contact & -1.6 & 1.7 & -465.1 & 27.7 & 43.4 & 4.3 & 930.1 & 55.5\\
Varying intercepts & -2.6 & 2.9 & -466.0 & 27.7 & 37.6 & 3.8 & 932.1 & 55.4\\
\bottomrule
\end{tabular}}
\end{table}

Table \ref{tab:modelcomparison1} shows the results of comparing three
models: a model with varying intercepts only, a model adding varying
slopes for spatial contacts, and a third model also adding the two
student-level predictors study phase and number of friendship
nominations received. The models are sorted from low to high LOOIC
values (corrected deviance) or, equivalently, from high to low values of
the model's \emph{expected log predictive density} (elpd) or expected
log-likelihood.

The model with the highest predictive accuracy according to the LOOIC
index (elpd\_loo), that is, with the lowest corrected deviance (LOOIC),
is shown in the top row of the table. In this application, the model
with varying effects of spatial contact and student-level predictors
performs best (LOOIC = 926.80). This model is compared to the other
models and the difference in predictive accuracy (elpd\_diff) is shown
in the first column. The model with varying slopes but without the
student-level predictors performs slightly worse (predictive accuracy
difference is -1.60), and the model without varying slopes and
student-level predictors performs worst (-2.60).

The sizes of the differences, however, are hard to interpret. If we want
to make sense of them, we should evaluate them in relation to the
standard errors of the predictive accuracy (se\_elpd\_loo). In this
example, the predictive accuracy differences are small in comparison to
the standard error of the predictive accuracy, so our uncertainty about
the predictive accuracy is much larger than the differences between the
models. Based on this information, no model clearly stands out as a
better or worse model.

\begin{wraptable}{r}{0.6\textwidth}

\caption{\label{tab:modelcomparison2}Model weights for predictive accuracy according to the approximate leave-one-out cross-validation information criterion (LOOIC) with stacking.}
\centering
\fontsize{8}{10}\selectfont
\begin{tabular}[t]{lr}
\toprule
Model & Weight\\
\midrule
Varying intercepts & 0.18\\
Plus varying slope spatial contact & 0.00\\
Plus student-level predictors & 0.82\\
\bottomrule
\end{tabular}
\end{wraptable}

Table \ref{tab:modelcomparison2} shows the weights for the three models
(Yao, Vehtari, Simpson, \& Gelman,
\protect\hyperlink{ref-YaoUsingStackingAverage2018}{2018}). The model
including student-level predictors clearly has the highest weight
(0.82). It has a much higher probability of yielding the best
predictions for new observations than the model with only varying
intercepts (0.18). The former probability is more than four times the
latter probability, so this is a substantial difference. Nevertheless,
the latter model has a substantial probability of yielding the best
predictions. The random slopes may slightly overfit the sample data for
which a model without random slopes corrects. So let us not discard the
variance components model. Instead, let us combine the two models,
weighting the model with their Akaike weights. The model with varying
effects but without student-level predictors can be ignored. It is very
unlikely to have the best predictions and it does not contribute to the
predictions in a model ensemble (model weight is 0.00).

\section{Conclusion}\label{conclusion}

Bayesian analyses are flexible and they provide results that are
intuitive.

\newpage

\section{References}\label{references}

\begingroup
\setlength{\parindent}{-0.5in} \setlength{\leftskip}{0.5in}

\hypertarget{refs}{}
\hypertarget{ref-AkaikeInformationtheoryextension1973}{}
Akaike, H. (1973). Information theory and an extension of the maximum
likelihood principle. In B. Petrov \& F. Cs\a'aki (Eds.), \emph{2nd
International Symposium on Information Theory, Tsahkadsor, Armenia,
USSR, eptember 2-8, 1971} (pp. 267--281). Budapest: Akadémiai Kiadó.

\hypertarget{ref-akaikeNewLookStatistical1974}{}
Akaike, H. (1974). A new look at the statistical model identification.
\emph{Institute of Electrical and Electronics Engineers. Transactions on
Automatic Control}, \emph{19}(6), 716--723.
doi:\href{https://doi.org/https://doi.org/10.1109/\%2FTAC.1974.1100705}{https://doi.org/10.1109\textbackslash{}\%2FTAC.1974.1100705}

\hypertarget{ref-bakanTestSignificancePsychological1966}{}
Bakan, D. (1966). The test of significance in psychological research.
\emph{Psychological Bulletin}, \emph{66}(6), 423--437.

\hypertarget{ref-R-lme4}{}
Bates, D., Maechler, M., Bolker, B., \& Walker, S. (2019). \emph{Lme4:
Linear mixed-effects models using 'eigen' and s4}. Retrieved from
\url{https://CRAN.R-project.org/package=lme4}

\hypertarget{ref-BerryStatisticsBayesianPerspective1996}{}
Berry, D. A. (1996). \emph{Statistics: A Bayesian Perspective}. Belmont,
CA: Duxbury Press.

\hypertarget{ref-BurnhamKullbackLeiblerinformationbasis2001}{}
Burnham, K. P., \& Anderson, D. R. (2001). Kullback-Leibler information
as a basis for strong inference in ecological studies. \emph{Wildlife
Research}, \emph{28}(2), 111--119.
doi:\href{https://doi.org/10.1071/wr99107}{10.1071/wr99107}

\hypertarget{ref-carverCaseStatisticalSignificance1978}{}
Carver, R. (1978). The Case Against Statistical Significance Testing.
\emph{Harvard Educational Review}, \emph{48}(3), 378--399.
doi:\href{https://doi.org/10.17763/haer.48.3.t490261645281841}{10.17763/haer.48.3.t490261645281841}

\hypertarget{ref-carverCaseStatisticalSignificance1993}{}
Carver, R. P. (1993). The Case against Statistical Significance Testing,
Revisited. \emph{The Journal of Experimental Education}, \emph{61}(4),
287--292.

\hypertarget{ref-ClarkBayesianBasicsConceptual2014}{}
Clark, M. (2014). \emph{Bayesian Basics. A Conceptual Introduction with
Application in R and Stan}. Center for Statistical Consultation and
Research, University of Michigan.

\hypertarget{ref-CummingUnderstandingnewstatistics2012}{}
Cumming, G. (2012). \emph{Understanding the new statistics: Effect
sizes, confidence intervals, and meta-analysis}. New York: Routledge.

\hypertarget{ref-depaoliImprovingTransparencyReplication2017}{}
Depaoli, S., \& van de Schoot, R. (2017). Improving transparency and
replication in Bayesian statistics: The WAMBS-Checklist.
\emph{Psychological Methods}, \emph{22}(2), 240--261.
doi:\href{https://doi.org/10.1037/met0000065}{10.1037/met0000065}

\hypertarget{ref-efronComputerAgeStatistical2016}{}
Efron, B., \& Hastie, T. (2016). \emph{Computer Age Statistical
Inference: Algorithms, Evidence, and Data Science} (1 edition.). New
York, NY: Cambridge University Press.

\hypertarget{ref-EricksonBeginningBayes2017}{}
Erickson, T. (2017). Beginning Bayes. \emph{Teaching Statistics},
\emph{39}(1), 30--35.
doi:\href{https://doi.org/10.1111/test.12121}{10.1111/test.12121}

\hypertarget{ref-R-shinystan}{}
Gabry, J. (2018). \emph{Shinystan: Interactive visual and numerical
diagnostics and posterior analysis for bayesian models}. Retrieved from
\url{https://CRAN.R-project.org/package=shinystan}

\hypertarget{ref-R-rstanarm}{}
Gabry, J., \& Goodrich, B. (2019). \emph{Rstanarm: Bayesian applied
regression modeling via stan}. Retrieved from
\url{https://CRAN.R-project.org/package=rstanarm}

\hypertarget{ref-GelmanBayesianDataAnalysis2013}{}
Gelman, A., Carlin, J. B., Stern, H. S., Dunson, D. B., Vehtari, A., \&
Rubin, D. B. (2013). \emph{Bayesian Data Analysis, Third Edition}. CRC
Press.

\hypertarget{ref-JamesIntroductionStatisticalLearning2017}{}
James, G., Witten, D., Hastie, T., \& Tibshirani, R. (2017). \emph{An
Introduction to Statistical Learning: With Applications in R} (1st ed.
2013, Corr. 7th printing 2017 edition.). New York: Springer.

\hypertarget{ref-kassBayesFactors1995}{}
Kass, R. E., \& Raftery, A. E. (1995). Bayes Factors. \emph{Journal of
the American Statistical Association}, \emph{90}(430), 773--795.
doi:\href{https://doi.org/10.2307/2291091}{10.2307/2291091}

\hypertarget{ref-kruschkeBayesianAssessmentNull2011}{}
Kruschke, J. K. (2011). Bayesian Assessment of Null Values Via Parameter
Estimation and Model Comparison. \emph{Perspectives on Psychological
Science}, \emph{6}(3), 299--312.
doi:\href{https://doi.org/10.1177/1745691611406925}{10.1177/1745691611406925}

\hypertarget{ref-kurtBayesianStatisticsFun2019}{}
Kurt, W. (2019). \emph{Bayesian Statistics the Fun Way}. San Francisco:
No Starch Press.

\hypertarget{ref-lambertStudentsGuideBayesian2018}{}
Lambert, B. (2018). \emph{A Students Guide to Bayesian Statistics} (1
edition.). Los Angeles: SAGE Publications Ltd.

\hypertarget{ref-leeBayesianStatisticsIntroduction2012}{}
Lee, P. M. (2012). \emph{Bayesian Statistics: An Introduction, 4th
Edition: An Introduction, 4th Edition} (4th edition.). Chichester, West
Sussex ; Hoboken, N.J: Wiley.

\hypertarget{ref-LevineQuantitativeCommunicationResearch2013}{}
Levine, T. R. (2013). Quantitative Communication Research: Review,
Trends, and Critique. \emph{Review of Communication Research}, \emph{1},
69--84. doi:\href{https://doi.org/10.12840}{10.12840}

\hypertarget{ref-LevineCriticalAssessmentNull2008}{}
Levine, T. R., Weber, R., Hullett, C., Park, H. S., \& Lindsey, L. L. M.
(2008). A Critical Assessment of Null Hypothesis Significance Testing in
Quantitative Communication Research. \emph{Human Communication
Research}, \emph{34}(2), 171--187.
doi:\href{https://doi.org/10.1111/j.1468-2958.2008.00317.x}{10.1111/j.1468-2958.2008.00317.x}

\hypertarget{ref-MadanSocialsensingepidemiological2010}{}
Madan, A., Cebrian, M., Lazer, D., \& Pentland, A. (2010). Social
sensing for epidemiological behavior change. In J. E. Bardram, M.
Langheinrich, K. N. Truong, \& P. Nixon (Eds.), \emph{Proceedings of the
12th ACM international conference on Ubiquitous computing} (pp.
291--300). ACM.

\hypertarget{ref-magnussonUsingLmeLmer2015}{}
Magnusson, K. (2015, April). Using R and lme/lmer to fit different two-
and three-level longitudinal models - R Psychologist. \emph{R
Psychologist}.

\hypertarget{ref-McElreathStatisticalRethinkingBayesian2015}{}
McElreath, R. (2015). \emph{Statistical Rethinking: A Bayesian Course
with Examples in R and Stan}. Boca Raton: CRC Press.

\hypertarget{ref-morrisonSignificanceTestControversy1970}{}
Morrison, D. E., \& Henkel, R. E. (1970). \emph{The Significance Test
Controversy: A Reader} (1st ed.). New York: Routledge.

\hypertarget{ref-TQMP14-2-99}{}
Muth, C., Oravecz, Z., \& Gabry, J. (2018). User-friendly Bayesian
regression modeling: A tutorial with rstanarm and shinystan. \emph{The
Quantitative Methods for Psychology}, \emph{14}(2), 99--119.
doi:\href{https://doi.org/10.20982/tqmp.14.2.p099}{10.20982/tqmp.14.2.p099}

\hypertarget{ref-tylerWhatStatisticalSignificance1931}{}
Tyler, R. W. (1931). What Is Statistical Significance? \emph{Educational
Research Bulletin}, \emph{10}(5), 115--142.

\hypertarget{ref-Vehtari2017}{}
Vehtari, A., Gelman, A., \& Gabry, J. (2017). Practical Bayesian model
evaluation using leave-one-out cross-validation and WAIC.
\emph{Statistics and Computing}, \emph{27}(5), 1413--1432.
doi:\href{https://doi.org/10.1007/s11222-016-9696-4}{10.1007/s11222-016-9696-4}

\hypertarget{ref-WagenmakersAICmodelselection2004}{}
Wagenmakers, E.-J., \& Farrell, S. (2004). AIC model selection using
Akaike weights. \emph{Psychonomic Bulletin \& Review}, \emph{11}(1),
192--196.
doi:\href{https://doi.org/10.3758/BF03206482}{10.3758/BF03206482}

\hypertarget{ref-WatanabeAsymptoticEquivalenceBayes2010}{}
Watanabe, S. (2010). Asymptotic Equivalence of Bayes Cross Validation
and Widely Applicable Information Criterion in Singular Learning Theory.
\emph{Journal of Machine Learning Research}, \emph{11}(Dec), 3571--3594.

\hypertarget{ref-YaoUsingStackingAverage2018}{}
Yao, Y., Vehtari, A., Simpson, D., \& Gelman, A. (2018). Using Stacking
to Average Bayesian Predictive Distributions. \emph{Bayesian Analysis}.
doi:\href{https://doi.org/10.1214/17-BA1091}{10.1214/17-BA1091}

\hypertarget{ref-ziliakCultStatisticalSignificance2008}{}
Ziliak, S. T., \& McCloskey, D. N. (2008). \emph{The cult of statistical
significance : How the standard error costs us jobs, justice, and
lives}. Ann Arbor, MI: University of Michigan Press.

\endgroup


\end{document}
